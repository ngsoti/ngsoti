\documentclass[10pt,a4paper]{report}

% Packages for formatting and styling
\usepackage{graphicx} % For including images
\usepackage{hyperref} % For hyperlinks and metadata
\usepackage{fancyhdr} % For headers and footers
\usepackage{geometry} % Adjusting page geometry
\usepackage{lastpage}
% Page geometry
\geometry{
    a4paper,
    top=2.5cm,
    bottom=2.5cm,
    left=2.5cm,
    right=2.5cm
}

% Define variables
\newcommand{\delivtitle}{D4.3 - NGSOTI training experience blog post}
\newcommand{\delivdate}{30 June 2025}

% Hyperref setup
\hypersetup{
    pdftitle={\delivtitle},
    pdfauthor={TEAM NGSOTI},
    pdfsubject={EU Project Deliverable Report},
    pdfkeywords={NGSOTI, EU Project, Deliverable, Cybersecurity},
    colorlinks=true,
    linkcolor=blue,
    urlcolor=blue
}

% Set up the header and footer
\pagestyle{fancy}
\fancyhf{} % Clear all header and footer fields
\fancyhead[L]{NGSOTI – Project: 101127921 —  DIGITAL-ECCC-2022-CYBER-03}
\fancyfoot[C]{\delivtitle{} - Page \thepage/\pageref{LastPage}}

% Title page setup
\title{
    \Huge \textbf{\delivtitle} \\[0.5cm]
    \includegraphics[width=0.3\textwidth]{img/ngsoti.eps}
    \hspace{1cm}
    \includegraphics[width=0.3\textwidth]{img/eu_funded_en.eps}
}
\author{\textbf{TEAM NGSOTI}}
\date{\delivdate}

% Document starts here
\begin{document}

% Title page
\maketitle
\thispagestyle{empty} % Remove header and footer from the title page

% Table of contents
\newpage
\tableofcontents
\newpage
\section*{Disclaimer}
Co-funded by the European Union. Views and opinions expressed are however
those of the author(s) only and do not necessarily reflect those of the
European Union or the European Cybersecurity Competence Centre. Neither
the European Union nor the granting authority can be held responsible for them.

\section*{Deliverable definition}
The identifier of the deliverable is D3.1 and it adheres to the definition
outlined in the grant agreement written in bold.
\textbf{Blogpost on training experience with NGSOTI how the tools were used on d4-project.org}

% Main content
\section*{Executive Summary}
\addcontentsline{toc}{section}{Executive Summary}
In the first 12 months of NGSOTI, \textbf{13 training sessions} were conducted, reaching
a total of \textbf{155 participants} across various sectors, including Multiple,
Education, and Finance. During these sessions, feedback was collected, and
the key findings are presented in this document. The first version was published on June 19th on the d4-project.org blog.

\section*{Our Experience and Key Takeaways from Trainins within the NGSOTI EU Funded Project}

We are glad to share the overwhelmingly positive outcomes and valuable
insights gained from our participation in the NGSOTI (Next Generation Security
Operations and Threat Intelligence) EU Funded Project. The project has been a
resounding success, allowing us to engage with a diverse audience and make
significant strides in cybersecurity education and training.

\subsection*{A Resounding Success in Numbers}
Throughout the project, we conducted 13 training sessions, reaching a total
of 155 participants. These participants hailed from a variety of professional
backgrounds, including the financial and academic sectors, creating a rich and
collaborative learning environment.

\subsection*{Tailored Training for the Financial Sector}
One of our key observations was the effectiveness of providing ready-to-use
training materials for professionals in the financial sector. Given the
fast-paced and highly regulated nature of their work, having access to
practical, immediately applicable knowledge and tools proved to be the most
efficient and valuable approach.

\subsection*{Bridging the Gap in Academia with Hands-On Tools}
In the academic sector, we identified a clear need for more practical,
hands-on tooling sessions for students. To address this, our training for
university students focused on Digital Forensics and Incident Response (DFIR)
and Threat Intelligence, utilizing cutting-edge open-source tools such as:

\begin{itemize}
    \item MISP - A Threat-Intelligence Platform
    \item AIL Project (Analysis Information Leak)
    \item Kunai
    \item FlowIntel
\end{itemize}

The engagement and feedback from students were exceptional, highlighting a
strong desire for this type of practical experience.

\subsection*{Rethinking University Curriculums: A Call for Practicality}

Our experience within university settings also brought to light several
shortcomings in traditional academic approaches to cybersecurity education:

\begin{description}
    \item {Lack of Practical Labs}: There is a significant scarcity of
hands-on laboratory sessions where students can apply theoretical knowledge.
    \item {Overemphasis on GRC}: Curriculums often focus heavily on Governance,
Risk, and Compliance (GRC) at the expense of practical, hands-on skills.
    \item {Onboarding Challenges}: We observed a learning curve for students
who were not accustomed to hands-on lab environments.
\end{description}

These observations point to a tremendous opportunity for universities to
rethink and modernize their cybersecurity curriculums. By incorporating more
practical, tool-based training, we can better prepare the next generation of
cybersecurity professionals for the real-world challenges they will face.

\subsection*{Conclusion}
The NGSOTI project has been a rewarding experience. It has not only allowed
us to share our expertise but has also provided us with invaluable insights
into the diverse needs of the cybersecurity community. We are more committed
than ever to championing hands-on, practical training and look forward to
continuing to bridge the gap between theoretical knowledge and real-world
application.
% End of document
\end{document}

