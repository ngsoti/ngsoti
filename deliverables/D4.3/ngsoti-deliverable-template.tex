\documentclass[10pt,a4paper]{report}

% Packages for formatting and styling
\usepackage{graphicx} % For including images
\usepackage{hyperref} % For hyperlinks and metadata
\usepackage{fancyhdr} % For headers and footers
\usepackage{geometry} % Adjusting page geometry
\usepackage{lastpage}
% Page geometry
\geometry{
    a4paper,
    top=2.5cm,
    bottom=2.5cm,
    left=2.5cm,
    right=2.5cm
}

% Define variables
\newcommand{\delivtitle}{D4.3 - NGSOTI training experience blog post}
\newcommand{\delivdate}{30 June 2025}

% Hyperref setup
\hypersetup{
    pdftitle={\delivtitle},
    pdfauthor={TEAM NGSOTI},
    pdfsubject={EU Project Deliverable Report},
    pdfkeywords={NGSOTI, EU Project, Deliverable, Cybersecurity},
    colorlinks=true,
    linkcolor=blue,
    urlcolor=blue
}

% Set up the header and footer
\pagestyle{fancy}
\fancyhf{} % Clear all header and footer fields
\fancyhead[L]{NGSOTI – Project: 101127921 —  DIGITAL-ECCC-2022-CYBER-03}
\fancyfoot[C]{\delivtitle{} - Page \thepage/\pageref{LastPage}}

% Title page setup
\title{
    \Huge \textbf{\delivtitle} \\[0.5cm]
    \includegraphics[width=0.3\textwidth]{img/ngsoti.eps}
    \hspace{1cm}
    \includegraphics[width=0.3\textwidth]{img/eu_funded_en.eps}
}
\author{\textbf{TEAM NGSOTI}}
\date{\delivdate}

% Document starts here
\begin{document}

% Title page
\maketitle
\thispagestyle{empty} % Remove header and footer from the title page

% Table of contents
\newpage
\tableofcontents
\newpage
\section*{Disclaimer}
Co-funded by the European Union. Views and opinions expressed are however
those of the author(s) only and do not necessarily reflect those of the
European Union or the European Cybersecurity Competence Centre. Neither
the European Union nor the granting authority can be held responsible for them.

\section*{Deliverable definition}
The identifier of the deliverable is D3.1 and it adheres to the definition
outlined in the grant agreement written in bold.
\textbf{Blogpost on training experience with NGSOTI how the tools were used on d4-project.org}

% Main content
\section*{Executive Summary}
\addcontentsline{toc}{section}{Executive Summary}
In the first 12 months of NGSOTI, 13 training sessions were conducted, reaching
a total of 155 participants across various sectors, including Multiple,
Education, and Finance. During these sessions, feedback was collected, and
the key findings are presented in this document.


\section*{NGOSTI: A Strategic Asset for Fulfilling Your Regulatory Duties under DORA, NIS, NIS2, and GDPR}

The Digital Operational Resilience Act (DORA) sets out critical requirements
for financial entities to strengthen their digital resilience, particularly in
managing ICT-related incidents. Among its key provisions, Article 17 requires
regulated entities to establish fundamental incident response capabilities,
including processes for detecting, managing, and notifying incidents in a
timely and structured manner.

Complementing this, Article 18 introduces the obligation to classify
ICT-related incidents based on impact and severity, ensuring consistency and
transparency in how disruptions are assessed and reported.

To support organizations in meeting these obligations, NGSOTI provides free
and publicly accessible training materials focused specifically on incident
detection and response. These resources are designed with practical application
in mind, incorporating real-world examples and data extracted from the NGSOTI
platform itself.

NGSOTI’s training offerings are flexible and accessible—available co-funded by
the European Union both as virtual sessions and in-person workshops—and aim
to build operational readiness for complying with DORA, as well as other
regulatory frameworks such as NIS, NIS2, and GDPR.

Whether you're just getting started or refining your incident response
strategy, NGSOTI’s training infrastructure is a valuable tool for ensuring
your organization stays compliant, resilient, and prepared.

NGSOTI training requests can be sent to info@circl.lu.


Besides the training material of incident response  NGOSTI provides tooling.
The section below shows how the tooling is used.

Besides the incident response training materials, NGSOTI also provides
dedicated tooling to support operational resilience.
The section below illustrates how the tooling can be used in various
operational contexts:

\begin{itemize}
    \item By local entities operating independently
    \item In a hybrid model combining local entities and contractual services
    \item Exclusively through contractual service providers
\end{itemize}
% End of document
\end{document}

