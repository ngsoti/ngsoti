\documentclass[10pt,a4paper]{report}

% Packages for formatting and styling
\usepackage{graphicx} % For including images
\usepackage{hyperref} % For hyperlinks and metadata
\usepackage{fancyhdr} % For headers and footers
\usepackage{geometry} % Adjusting page geometry
\usepackage{lastpage}
% Page geometry
\geometry{
    a4paper,
    top=2.5cm,
    bottom=2.5cm,
    left=2.5cm,
    right=2.5cm
}

% Define variables
\newcommand{\delivtitle}{D3.1 - References of training material updates \#1}
\newcommand{\delivdate}{31 March 2025}

% Hyperref setup
\hypersetup{
    pdftitle={\delivtitle},
    pdfauthor={TEAM NGSOTI},
    pdfsubject={EU Project Deliverable Report},
    pdfkeywords={NGSOTI, EU Project, Deliverable, Cybersecurity},
    colorlinks=true,
    linkcolor=blue,
    urlcolor=blue
}

% Set up the header and footer
\pagestyle{fancy}
\fancyhf{} % Clear all header and footer fields
\fancyhead[L]{NGSOTI – Project: 101127921 —  DIGITAL-ECCC-2022-CYBER-03}
\fancyfoot[C]{\delivtitle{} - Page \thepage/\pageref{LastPage}}

% Title page setup
\title{
    \Huge \textbf{\delivtitle} \\[0.5cm]
    \includegraphics[width=0.3\textwidth]{img/ngsoti.eps}
    \hspace{1cm}
    \includegraphics[width=0.3\textwidth]{img/eu_funded_en.eps}
}
\author{\textbf{TEAM NGSOTI}}
\date{\delivdate}

% Document starts here
\begin{document}

% Title page
\maketitle
\thispagestyle{empty} % Remove header and footer from the title page

% Table of contents
\newpage
\tableofcontents
\newpage
\section*{Disclaimer}
Co-funded by the European Union. Views and opinions expressed are however those of the author(s) only
and do not necessarily reflect those of the European Union or the European Cybersecurity Competence Centre. Neither
the European Union nor the granting authority can be held responsible for them.
% Main content

\section*{Deliverable definition}
The identifier of the deliverable is D3.1 and it adheres to the definition outlined in the grant agreement
written in bold. \textbf{A list of commits of in public training material git repositories}

\section*{Executive Summary}
In the NGOSTI project, new training materials are developed, or existing ones
are updated. Some of these materials are released under an open-source license,
allowing multiple contributors from various projects to enhance and extend them.
This report focuses on NGOSTI training programs in the following three domains:

\begin{itemize}
    \item Incident Response
    \item MISP
    \item Cryptography
\end{itemize}

This report includes references to commits in the public training material
repositories.

\chapter{Training Material}

\section{Incident Response}

NGOSTI Incident Reponse training is tailored to each organization receiving the
training. Thus it includes many sensitive data about their internal working
organization and infrastructure.
A generic training was distilled and released on the NGOSTI project for this
deliverable\footnote{\url{https://github.com/ngsoti/ngsoti/tree/main/training/incident-response/NGSOTI-Incident-Response}}
It consists in 71 slides of training material.

\section{Cryptography}

Cryptography Concepts - Past and Present was designed and delivered by Georges Kesseler, IT sysadmin \& Course Manager at Digital Learning Hub of Luxembourg (DLH) as part of the NGOSTI training program for Master students. The course introduces historical and modern cryptographic methods, including substitution ciphers (Caesar, Pigpen), polyalphabetic ciphers (Vigenère), mechanical encryption (Enigma), and contemporary cryptosystems (RSA, Elliptic Curves, symmetric and asymmetric encryption). It combines conceptual foundations with practical exercises on the use of cryptography in securing digital communication.

The training materials are available on the NGSOTI Github repository for this deliverable\footnote{\url{https://github.com/ngsoti/ngsoti/tree/main/training/cryptography/NGSOTI-DLH-Cryptography}}. They consist of training slides and three exercise worksheets for enryption and decryption.



The list of commits related to trainings is shown in Table ~\ref{ir}.


\begin{table}[h]
\begin{tabular}{llll}
    Date & Repository & Title & commit id \\
    \hline
    2025-03-19 & NGSOTI & Introduction to Incident Response & e4691c02f8598bde45d78b11076a3c72dfbf5a02\\
    2025-03-20 & NGSOTI & Cryptography - Past and Present & 10d49a74ee373709d697d70cd3b285dcb03b7bee\\

\end{tabular}
\caption{Git contributions to public training materials for Incident Response and Cryptography trainings}
\label{ir}
\end{table}

% End of document
\end{document}

