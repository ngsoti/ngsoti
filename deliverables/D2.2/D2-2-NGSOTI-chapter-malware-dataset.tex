\chapter{Malware Dataset}

In the context of the NGSOTI project, we are actively growing and maintaining a malware samples dataset\footnote{\href{https://helga.circl.lu/NGSOTI/malware-dataset}{https://helga.circl.lu/NGSOTI/malware-dataset}}. This initiative does not aim to create yet another malware-sharing repository. Instead, its primary objective is to host representative samples from various malware families.

Alongside the malware samples, the dataset also includes analysis data—such as packet capture (PCAP) files, log files, and other artifacts—extracted through automated analysis. These analyses leverage tools developed within the NGSOTI project, including Kunai\footnote{\href{https://github.com/kunai-project/kunai}{https://github.com/kunai-project/kunai}}. This dataset serves as a foundational resource for further activities and research within the project.

The following are the key applications we plan for this data:

\begin{itemize}
	\item Rule Development and Sharing: Leverage Kunai logs to create and share detection rules with the community\footnote{\href{https://github.com/kunai-project/community-rules}{https://github.com/kunai-project/community-rules}}. These rules aim to enhance threat detection and provide actionable intelligence.
	\item SOC Training Scenarios: Use Kunai logs and PCAP files as realistic injects for Security Operations Center (SOC) training exercises. This enables the simulation of real-world attack scenarios to enhance the skills of SOC analysts.
\end{itemize}


